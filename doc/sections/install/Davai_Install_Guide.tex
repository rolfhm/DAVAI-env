\documentclass[a4paper,10pt,twoside]{article}

\usepackage[utf8]{inputenc}
\usepackage{enumitem}
\usepackage{moreenum}
\usepackage{graphicx}
\usepackage{fancyref}
\usepackage{multirow}
\usepackage{hyperref}
\usepackage[normalem]{ulem}
\usepackage[title]{appendix}
\usepackage{listings}

\textheight25.0cm
\textwidth16cm
\topmargin-.5cm
\oddsidemargin 0.cm
\evensidemargin 0.cm
\headheight0.5cm
\headsep0.cm
\footskip0.7cm
%\footheight2.5cm
\marginparwidth1.2cm
\marginparsep0.3cm

%%%%%%%%%%%%%%%%%%%%%%%%%%%%%%%%%%%%%%%%%%%%%%%%% END of HEADER

\title{DAVAÏ Install Guide}
\author{A. Mary}
\date{Apr. 2021}

\pdfinfo{%
  /Title    (Ugr)
  %/Author   (A. mary)
  /Creator  ()
  /Producer ()
  /Subject  ()
  /Keywords ()
}

\begin{document}
\maketitle

This document describes DAVAÏ install procedures for a new user. On platforms where it is already ported, this procedure is much easier. On other platforms, there are a number of additional steps. 
Specific \textit{porting} steps can also be necessary on other platforms to adapt some of the software to the new environment.

\tableofcontents
\vspace{1cm}
\newpage






\section{On pre-installed platforms}

\subsection{MF supercomputer \texttt{belenos}}
This section describes the step-by-step initialization for a user on Météo France HPC cluster \texttt{\textit{belenos}}, where the environment is already pre-installed.

\subsubsection{Set your IFS-Arpege-LAM (IAL) repository}
At time of writing this documentation, the ACCORD forge for the IAL [\textbf{I}FS-\textbf{A}rpege-\textbf{L}AM] repository is not available yet, and therefore we still use the legacy GCO Git repository for the publishing of releases, updates of integration branches, and posting of contributions.
This repository is hosted on the machine \textit{\texttt{mirage7}} within Météo France internal network.

However, so far, no direct Git connexion is possible between the \textit{\texttt{belenos}} HPC and the \textit{\texttt{mirage7}} machine.
Therefore, and until the ACCORD forge is available, we use a proxy through a relay machine to connect your \textit{\texttt{belenos}} repository to the GCO official repository on \textit{\texttt{mirage7}}.
The relay machine ought to be one of the \texttt{\textit{sotrtm3?-sidev}} servers within MF.

In the following, the relay machine will be refered to as \texttt{\textit{\_\_relay\_\_}}, to be replaced in the command lines by the actual machine you have an account on (e.g. \textit{\texttt{sotrtm31-sidev}}). If you don't have an account on one of these machines, please contact Eric Escalière.\\

All of these constraints make this step a bit more complicated than it should be, but it will become easier with the setting of an IAL forge.\\

\noindent \textbf{Note:} the GCO-Git toolbox is still in use, for now, at least \underline{for initializing} the repository, \underline{creating} a branch and \underline{posting} it. Native \texttt{git push} to \texttt{\textit{mirage7}} server is not possible.

\subsubsection*{To configure your repository this way on \texttt{\textit{belenos}}:}
\begin{enumerate}[label=(IAL.\arabic*)]
 \item Ensure SSH configuration between \textit{\texttt{belenos}} and \textit{\texttt{\_\_relay\_\_}} machine:\\
 set your \textit{\texttt{belenos}} SSH public key (\texttt{belenos:$\sim$/.ssh/id\_rsa.pub}) in your authorized keys on the \textit{\texttt{\_\_relay\_\_}} machine (\texttt{\textit{\_\_relay\_\_}:$\sim$/.ssh/authorized\_keys}), and conversely.
 \item Set your Git and GCO-Git environment (these steps are welcome in your \texttt{$\sim$/.bash\_profile}):
 \begin{itemize}
  \item set python to Python3 by default:\\
  \texttt{module load python}
  \item load a recent version of Git:\\
  \texttt{module load git}
  \item load GCO-Git toolbox:\\
  \texttt{export PATH=/home/mf/dp/marp/gco/apps/git-toolbox/client/default/bin:\$PATH}
  \item and set GCO-Git toolbox configuration:
  \begin{itemize}
   \item choose a location for your repository, e.g.:\\
   \texttt{export GIT\_HOMEPACK=\$HOME/repositories} (directory must exist)
   \item configure the relay mechanism:\\
   \texttt{export GIT\_SERVER\_UPDATE\_METHOD=ssh}\\
   \texttt{export GIT\_SERVER\_UPDATE\_PROXY=\textit{\_\_relay\_\_}}
   \item specify the origin repository:\\
   \texttt{export GIT\_ROOTPACK=\$GIT\_SERVER\_UPDATE\_PROXY:/home/marp/martinezs/git-rootpack}
   \item[!] \textit{(Beware of the \_ if you copy-past these commands !)}
  \end{itemize}
 \end{itemize}
 \item Initialize your repository (should take a little while):\\
 \texttt{git\_start}
\end{enumerate}


\subsubsection{Dependencies}
Specific packages used by DAVAÏ are already installed on \textit{\texttt{belenos}} and referenced in config files, so nothing is to be done for these.

\subsubsection{Install DAVAÏ environment and set your preferences}

$\hookrightarrow$ see generic procedure here: \ref{sect:davai_install}

\subsection{MF workstations}
(tbc)

\subsection{ECMWF supercomputers}
(tbc)






\newpage
\section{General installation}
\subsection{Dependencies}

DAVAÏ works over a number of NWP packages, tools, software, that need to be installed on the machine with their own procedures.
These include:
\begin{itemize}
 \item \textit{\textbf{Vortex}}: scripting system used for the definition of tasks and running of jobs\\
 \href{https://opensource.umr-cnrm.fr/projects/vortex}{https://opensource.umr-cnrm.fr/projects/vortex}
 \item \textit{\textbf{IAL-expertise}}: tools to analyse automatically the outputs of NWP configurations -- norms, Jo-tables, fields in FA/GRIB files, ...\\
 \href{https://github.com/ACCORD-NWP/IAL-expertise}{https://github.com/ACCORD-NWP/IAL-expertise}
 \item \textit{\textbf{IAL-build}}: wrappers around \texttt{git} and \texttt{gmkpack} to build executables from specified Git references\\
 \href{https://github.com/ACCORD-NWP/IAL-build}{https://github.com/ACCORD-NWP/IAL-build}
 \item ...
 \item (not yet) \texttt{ecbundle}: \href{https://git.ecmwf.int/projects/ECSDK/repos/ecbundle}{https://git.ecmwf.int/projects/ECSDK/repos/ecbundle}
\end{itemize}


\subsection{Set your IFS-Arpege-LAM (IAL) repository}
To be cloned from \texttt{github.com/ACCORD-NWP/IAL.git} when available...

%If you don't have a local clone of the IAL repository, clone it to e.g. \texttt{$\sim$/repositories}:
%\begin{itemize}
% \item \texttt{cd $\sim$/repositories}
% \item \texttt{git clone https://github.com/ACCORD-NWP/IAL.git}
%\end{itemize}


\subsection{Install DAVAÏ environment and set your preferences \label{sect:davai_install}}

The tests templates and config files are versioned using Git, so that we can have different versions of the tests according to the version of the IAL that we are testing. The user therefore has to clone the \texttt{DAVAI-env} repository (rather than copy) to be able to switch from a version to another.\\

The \texttt{DAVAI-env} repository contains the DAVAÏ command-lines, some inner utilities, the documentation, and the nested repository \texttt{DAVAI-tests}. This latter contains the tests templates \& wrappers and config files, and it is versioned innerly to the nested repository, because changes to the tests are more frequent than to the rest of \texttt{DAVAI-env}.\\

\noindent To install it:
\begin{enumerate}[label=(D.\arabic*)]
 \item go to the directory where you want to install it (e.g. \texttt{$\sim$/repositories}) and\\
       clone the \texttt{DAVAI-env} repository:\\
       \texttt{git clone https://github.com/ACCORD-NWP/DAVAI-env.git}\\
       then go in the newly cloned directory \texttt{DAVAI-env}
 \item if you want to customize your install and default values for DAVAÏ options and packages, you can do so following the instructions given by:\\
       \texttt{davai-init -s}\\
       In particular, you can set the path to your IAL repository in the user config.
 \item initialize your environment\footnote{initialize the nested repository, set user directories and user config.} :\\
       \texttt{bin/davai-init}\\
       (or \texttt{make init})\\
       and export PATH as suggested by prompt (and/or set in \texttt{.bash\_profile})
\end{enumerate}







\newpage
\section{Porting to a new platform: adaptations}
(tbc)

\end{document}
